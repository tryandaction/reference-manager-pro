% 模拟真实的 LaTeX 论文
\documentclass[12pt]{article}
\usepackage[utf8]{inputenc}
\usepackage{cite}
\usepackage{amsmath}
\usepackage{graphicx}

\title{深度学习在图像识别中的应用研究}
\author{张三}
\date{\today}

\begin{document}

\maketitle

\begin{abstract}
本文综述了深度学习在图像识别领域的最新进展。我们回顾了卷积神经网络的发展历程,
并讨论了 ResNet \cite{he2016deep}、VGG \cite{simonyan2014very} 等经典架构的贡献。
\end{abstract}

\section{引言}

深度学习近年来取得了巨大突破 \cite{lecun2015deep}。
特别是在计算机视觉领域,卷积神经网络(CNN)已成为主流方法 \cite{krizhevsky2012imagenet}。

\section{相关工作}

\subsection{卷积神经网络}

AlexNet \cite{krizhevsky2012imagenet} 在 2012 年 ImageNet 竞赛中取得突破性成果。
随后,VGGNet \cite{simonyan2014very} 证明了网络深度的重要性。
ResNet \cite{he2016deep} 通过残差连接解决了深层网络的训练问题。

\subsection{注意力机制}

Transformer 架构 \cite{vaswani2017attention} 引入了自注意力机制,
后来被应用于视觉任务,产生了 Vision Transformer (ViT) \cite{dosovitskiy2020image}。

\section{方法}

我们的方法基于 \cite{he2016deep} 的残差学习框架,
结合了 \cite{vaswani2017attention} 的注意力机制。

\section{实验}

实验在 ImageNet 数据集上进行 \cite{deng2009imagenet}。
我们使用 PyTorch 框架 \cite{paszke2019pytorch} 实现模型。

\section{结论}

本文展示了深度学习在图像识别中的有效性。
未来工作将探索更高效的架构设计。

\bibliographystyle{plain}
\bibliography{references}

\end{document}
